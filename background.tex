\documentclass{article}
\usepackage{hyperref}
\usepackage{graphicx}
\newtheorem{definition}{Definition}


\begin{document}


\section{Background}
SAT problem is about satisfiability of Boolean formulas in conjunctive normal form (CNF). The necessary definitions for this can be read below. The definitions are valid in propositional logic.
\subsection{Boolean formulas and satisfiability}
\begin{definition}
In propositional logic we use symbols to represent propositions, these symbols called propositional variable which can be true or false.
\end{definition}
\begin{definition}
Interpretation is a function which maps propositional variables to logical values (true or false). In other words, the interpretation determines whether the value of a variable is true or false).
\end{definition}
\begin{definition}
\begin{enumerate}
    \item Every propositional variable is a formula.
    \item If $X$ is a formula, then $\neg X$ is a formula too, where $\neg $ is the sign of negation.
    \item If $X$ and $Y$ is a formula, then $(X   \circ Y$) is a formula too, where $\circ$ can be $\wedge$ (conjunction) or $\vee$ (disjunction) or $\supset$ (implication).
\end{enumerate}
Every formula is created by using the previous rules finitely many times.
\end{definition}
\begin{definition}
An interpretation satisfies a formula if the formula is true in that interpretation.
\end{definition}
\begin{definition}
A formula is satisfiable, if exists an interpretation which satisfies it, otherwise it is unsatisfiable.
\end{definition}
\subsection{Conjunctive normal form}
\begin{definition}
A propositional variable ($X$) and its negated form ($\neg X$) are called literals.
\end{definition}
\begin{definition}
A disjunction of literals is called clause.
\end{definition}
\begin{definition}
A conjunction of literals is a formula is in conjunctive normal form (CNF).
\end{definition}


\end{document}