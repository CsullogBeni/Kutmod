\documentclass{article}
\usepackage{hyperref}
\usepackage{graphicx}


\begin{document}


\section{Related works}

\subsection{Stochastic Boolean Satisfiability}

The article of Stochastic Boolean Satisfiability presents a connection point between a satisfiability problem and a probabilistic model. It Demonstrates how to adapt a view of satisfiability on the field of probability. SSAT in focus, a general stochastic satisfiability problem, which plays a similar role in probabilistic domains like SAT in deterministic ones.The article analyze the relationship between planning and uncertainty. Shows systematic and stochastic algorithms, for SSAT solutions. Perhaps the complexity gap between SSAT (PSPACE) and SAT (NP), is large, the article suggests, that the knowledge of SAT could be applied to the probabilistic domain.

\subsection{A survey of SAT solver}

In Computer Science, the Boolean Satisfiability Problem(SAT) is the problem of determining if there exists an interpretation that satisfies a given Boolean formula. SAT is one of the first problems that was proven to be NP-complete, which is also fundamental to artificial intelligence, algorithm and hardware design. This paper reviews the main algorithms of the SAT solver in recent years, including serial SAT algorithms, parallel SAT algorithms, SAT algorithms based on GPU, and SAT algorithms based on FPGA. The development of SAT is analyzed comprehensively in this paper. Finally, several possible directions for the development of the SAT problem are proposed.

\subsection{Sequential SAT Solvers}

The first SAT Solvers were sequential ones. These based on the Davis-Putnam-Loveland-Logemann (DPLL) algorithm. This algorithm includes rules which help to generate and traverse a binary search tree. Each part of the search tree is equal to a partial interpretation. The tree's branch elements walked through by a back-track algorithm and the main goal is to explore recent alternatives. This led to the CDCL (Conflict Driven Clause Learning) algorithm. The algorithm resolve the conflict clause and those clauses which were used in the implications, and learn these ones. After, these clauses can be added to the further formulas and improve the backtracking behaviour. Another technique was the "restarts", which helped to upgrade the performance. The most common SAT Solver which combine the CDCL and restart technique is the MINI-SAT. The MINI-SAT was introduced in 2003 and got improvements in each year and provided a basic schema, for sequential and parallel SAT-Solvers.


\subsection{An overview of parallel SAT solving}

In recent years the multicore architectures are becoming more and more wide\-spread. The SAT solvers should take advantage of this. Therefore, researchers have put a lot of focus on parallel algorithms. In the An overview of parallel SAT solving article we can read a great summary of the results in this field. The paper demonstrates the search space splitting and portfolio strategies which are the most important approaches. Furthermore, it contains some other significant solutions, for example hybrid methods. The different techniques present how multicore architectures can be used in SAT solving, and these had a relevant impact on our research.


\subsection{Parallel SAT Solvers}

At the dawn of the Parallel SAT Solvers, the first Solver only used single-core CPU, and communicated via network. When the memory sharing architectures became available, the scientists made studies on which is faster and supplemented the Parallel SAT Solver with more CPU cores and memory. The first impressions were really great, but later they found out, that if you use too many CPU cores and memory sharing, the efficiency decreases, because of the latency with the shared memory parts and the search for the optimal core slows down the software. The root of the problem was that the solver could not handle the underlying hardware very well and the calculations on the other cores was not perfectly balanced. So this helped to identify the principles what a many-core SAT Solver should achieve, in order to be an efficient Parallel SAT Solver:
1. The solver should aware of the hardware components in order to utilize the non-uniform access.
2. The solver should exchange the learned clauses and the data between the processes which running on different cores and has to be balanced well from the calculus up to the hardware level.
So lastly, if you want to create a well designed parallel SAT Solver, you must have a redesigned and reviewed Sequential Solver. It is still an open question that how you can build an efficient SAT Solver with a high level of scalability, combined with the modern multi-core architecture.
Our work is not mainly aiming this part of the SAT Solvers, but we made the first step towards a more efficient and more scalable parallel SAT Solver. The next topic will explain the details of our progress.

\section{References}

[1] - \href{https://link.springer.com/article/10.1023/A:1017584715408}{Stochastic Boolean Satisfiability}
[2] - \href{https://doi.org/10.1063/1.4981999}{A survey of SAT solver}
[3] - \href{https://link.springer.com/article/10.1007/s10601-012-9121-3}{An overview of parallel SAT solving}
[4] - \href{https://www.researchgate.net/publication/254048189_A_short_overview_on_modern_parallel_SAT-solvers}{A short overview on modern parallel SAT Solvers}

\end{document}