\documentclass{article}
\usepackage{hyperref}
\usepackage{graphicx}


\begin{document}


\section{Related works}

\subsection{Stochastic Boolean Satisfiability}

The article of Stochastic Boolean Satisfiability presents a connection point between a satisfiability problem and a probabilistic model. It Demonstrates how to adapt a view of satisfiability on the field of probability. SSAT in focus, a general stochastic satisfiability problem, which plays a similar role in probabilistic domains like SAT in deterministic ones.The article analyze the relationship between planning and uncertainty. Shows systematic and stochastic algorithms, for SSAT solutions. Perhaps the complexity gap between SSAT (PSPACE) and SAT (NP), is large, the article suggests, that the knowledge of SAT could be applied to the probabilistic domain.

\subsection{A survey of SAT solver}

In Computer Science, the Boolean Satisfiability Problem(SAT) is the problem of determining if there exists an interpretation that satisfies a given Boolean formula. SAT is one of the first problems that was proven to be NP-complete, which is also fundamental to artificial intelligence, algorithm and hardware design. This paper reviews the main algorithms of the SAT solver in recent years, including serial SAT algorithms, parallel SAT algorithms, SAT algorithms based on GPU, and SAT algorithms based on FPGA. The development of SAT is analyzed comprehensively in this paper. Finally, several possible directions for the development of the SAT problem are proposed.

\subsection{An overview of parallel SAT solving}

In recent years the multicore architectures are becoming more and more wide\-spread. The SAT solvers should take advantage of this. Therefore, researchers have put a lot of focus on parallel algorithms. In the An overview of parallel SAT solving article we can read a great summary of the results in this field. The paper demonstrates the search space splitting and portfolio strategies which are the most important approaches. Furthermore, it contains some other significant solutions, for example hybrid methods. The different techniques present how multicore architectures can be used in SAT solving, and these had a relevant impact on our research.

\section{References}

[1] - \href{https://link.springer.com/article/10.1023/A:1017584715408}{Stochastic Boolean Satisfiability}
[2] - \href{https://doi.org/10.1063/1.4981999}{A survey of SAT solver}
[3] - \href{https://link.springer.com/article/10.1007/s10601-012-9121-3}{An overview of parallel SAT solving}

\end{document}